%                                                                 aa.dem
% AA vers. 8.2, LaTeX class for Astronomy & Astrophysics
% demonstration file
%                                                       (c) EDP Sciences
%-----------------------------------------------------------------------
%
%\documentclass[referee]{aa} % for a referee version
%\documentclass[onecolumn]{aa} % for a paper on 1 column  
%\documentclass[longauth]{aa} % for the long lists of affiliations 
%\documentclass[rnote]{aa} % for the research notes
%\documentclass[letter]{aa} % for the letters 
%\documentclass[bibyear]{aa} % if the references are not structured 
% according to the author-year natbib style

%
\documentclass[]{aa}

%
\usepackage{graphicx}
\usepackage{xcolor}
\usepackage{natbib}
\bibpunct{(}{)}{;}{a}{}{,} % to follow the A&A style
%%%%%%%%%%%%%%%%%%%%%%%%%%%%%%%%%%%%%%%%
\usepackage{txfonts}
%%%%%%%%%%%%%%%%%%%%%%%%%%%%%%%%%%%%%%%%
\usepackage{hyperref}
% To add links in your PDF file, use the package "hyperref"
% with options according to your LaTeX or PDFLaTeX drivers.
%
\begin{document} 


   \title{The GALAH Survey: The mass assembly of the Milky Way as reconstructed from integrated GALAH spectra}
   \titlerunning{The mass assembly of the Milky Way with GALAH}


   \author{The GALAH collaboration, including S. Buder\inst{1}, A. Boecker\inst{1}}

   \authorrunning{The GALAH collaboration}

   \institute{Max Planck Institute for Astronomy (MPIA), Koenigstuhl 17, D-69117 Heidelberg}

   \date{Received XX XX, 2019; accepted XX XX, 2019}
 
  \abstract{We use the method developed by \cite{Boecker2019} to estimate the mass assembly of our Milky Way with integrated spectra from the GALAH survey. By mimicking an observation of stellar spectra from an arbitrary point (far away from the Milky Way stellar disk, we can treat the Milky Way like an extragalactic object for which the method by \cite{Boecker2019}. We confirm the inferred age-metallicity-relation because of the individually available stellar ages and metallicities from stellar spectroscopy \citep{Buder2019b}. We infer the mass of the most massive accreted galaxy onto the Milky Way and compare it with Sagittarius dSph as well as \textit{Gaia}-Enceladus and find that X.
}

   \keywords{}

   \maketitle
%
%________________________________________________________________

%________________________________________________________________
\section{Introduction}

%________________________________________________________________
\section{Data selection} \label{sec:selection}

We take spectra from the GALAH survey \citep{deSilva2015} with estimated radial velocities \citep{Buder2019b} and 5D information from \textit{Gaia} DR2 \citep{Brown2018}. With \textsc{galpy} \citep{Bovy2015} coordinate transformations to the Galactocentric coordinate system, we infer i.a. the vertical velocity $v_Z$.

\subsection{Mimicking the observation of extragalactic objects}

The method by \cite{Boecker2019} has been used for objects that are situated outside of the Galactic disk, such as Globular Clusters or extragalactic objects.

When we want to apply this method to Milky Way stars, observed within the disk, we have to take into account the influence of the line-of-sight velocities and adjust them to mimic an observation of MW stars from outside the MW.

To mimic the observation of the Milky Way as an extragalactic object, we simply assume that we observe the Milky way from a position $XYZ = (0,0,1000)\,\mathrm{kpc}$, so that the line-of-sight velocity is similar to the vertical velocity, $v_Z$ (to first order). With this Ansatz, we hence only have to shift the observed stellar spectra to the rest frame and then apply a velocity shift with the velocity with respect to the Galactic height, $v_Z$ and then integrate them.

%
%________________________________________________________________
%\section{Discussion}

%
%________________________________________________________________
%\section{Conclusions}


%-------------------------------------------------------------------

% for the bibliography, at the end
\bibstyle{aa} % style aa.bst
\bibliographystyle{aa} % style aa.bst
\bibliography{bib.bib} % your references Yourfile.bib

\label{LastPage}
\end{document}